%++++++++++++++++++++++++++++++++++++++++
% Don't modify this section unless you know what you're doing!
\documentclass[article,11pt]{article}
\usepackage{tabularx} % extra features for tabular environment
\usepackage{amsmath}  % improve math presentation
\usepackage{graphicx} % takes care of graphic including machinery
\usepackage[margin=1in,letterpaper]{geometry} % decreases margins
\usepackage{cite} % takes care of citations
\usepackage[final]{hyperref} % adds hyper links inside the generated pdf file
\hypersetup{
	colorlinks=true,       % false: boxed links; true: colored links
	linkcolor=blue,        % color of internal links
	citecolor=blue,        % color of links to bibliography
	filecolor=magenta,     % color of file links
	urlcolor=blue         
}
%++++++++++++++++++++++++++++++++++++++++


\begin{document}
	
	\title{Reinforcement Learning Project\\Learn Atari game \textit{Gopher} through Deep Reinforcement Learning}
	\author{Ivan Bergonzani}
	\date{\today}
	\maketitle
	
	\begin{abstract}
		In this project were tested different deep Q network architectures against the Atari 2600 game 'Gopher'.
		Base Deep Q network from \cite{dqn2013}\cite{dqn2015} together with Double Q network \cite{doubledqn} and Dueling DQN \cite{duelingdqn} were trained on the environment provided by OpenAI Gym for a total of 2 million frames each. Despite the smaller training time with respect to the original articles, the three network were able to learn the game. They were tested over a 1000 epsiodes scoring respectively a mean reward of 150, 152, 1521.
		
	\end{abstract}
	
	
	\section{Introduction}
	
	The very important physical effect has applications to astronomy, nuclear physics, condensed matter, and more. 
	

	\section{Deep Reinforcement Learning}

	\subsection{Double Q Learning}
	
	\subsection{Dueling Q network}
	
	\section{Experiments}
	
	\section{Conclusion}
	
	%++++++++++++++++++++++++++++++++++++++++
	% References section will be created automatically 
	% with inclusion of "thebibliography" environment
	% as it shown below. See text starting with line
	% \begin{thebibliography}{99}
	% Note: with this approach it is YOUR responsibility to put them in order
	% of appearance.
	
	\bibliography{bibliography}
	\bibliographystyle{plain}
\end{document}
